\documentclass[dvipdfmx]{jsarticle}

\usepackage{amsfonts}
\usepackage{amsmath, amssymb, bm}
\usepackage{natbib}
\usepackage{graphicx}
\usepackage{url}
\usepackage{breqn}
\usepackage{tikz}
\usepackage{here}

\newcommand{\argmax}{\mathop{\rm arg~max}\limits}
\newcommand{\argmin}{\mathop{\rm arg~min}\limits}

\title{UnUniFi}

\author{KIMURa Yu, SHIMOJIMA Takeru}

\begin{document}
\numberwithin{equation}{section}

\maketitle

\section{NFTFi}

\subsection{Collateral Liquidation Deposit Auction}

\subsubsection{Definitions}

\begin{itemize}
  \item[$i \in I$] index of bids
  \item[$n = |I|$] number of bids
  \item[$\{d_i\}_{i \in I}$] the deposit amount of $i$ th bid
  \item[$\{r_i\}_{i \in I}$] the interest rate of $i$ th bid (not annualized. for the period between the start and expiration date)
  \item[$\{x_i\}_{i \in I}$] the expiration date of $i$ th bid
  \item[$t$] current time 
  \item[$q$] the average of $p_i$ and the upper bound of $s$
  \item[$s$] the sum of deposits and the amount which lister can borrow with NFT as collateral
  \item[$\{a_i\}_{i \in I}$] means the amount borrowed from $i$ th bid deposit
  \item[$b$] the sum of $a_i$
  \item[$i_p(j)$] the index of the $j$ th highest price bid
  \item[$i_d(j)$] the index of the $j$ th highest deposit amount bid
  \item[$i_r(j)$] the index of the $j$ th lowest interest rate bid
  \item[$i_t(j)$] the index of the $j$ th farthest expiration date bid
  \item[$c$] minimum deposit rate
\end{itemize}

$$
  q = \frac{1}{n} \sum_{i \in I} p_i
$$

$$
  s = \sum_{i \in I} d_i
$$

$$
  b = \sum_{i \in I} a_i
$$

\subsubsection{State transition}

When $(p_{\text{new}}, d_{\text{new}}, r_{\text{new}}, t_{\text{new}})$ will be added to the set of bids, the new bids sequence will be

\begin{itemize}
  \item $I' = I \cup \{n+1\}$
  \item $n' = n + 1$
  \item $\{p_i'\}_{i \in I'} = \{p_i\}_{i \in I} \cup \{p_{\text{new}}\}$
  \item $\{d_i'\}_{i \in I'} = \{d_i\}_{i \in I} \cup \{d_{\text{new}}\}$
  \item $\{r_i'\}_{i \in I'} = \{r_i\}_{i \in I} \cup \{r_{\text{new}}\}$
  \item $\{t_i'\}_{i \in I'} = \{t_i\}_{i \in I} \cup \{t_{\text{new}}\}$
  \item $q' = \frac{1}{n'} \sum_{i \in I'} p_i'$
  \item $s' = \sum_{i \in I'} d_i'$
\end{itemize}

where the prime means the next state.

The constraint of $d_{\text{new}}$ is

$$
  c p_{\text{new}} \le d_{\text{new}} \le q' - s
$$

In easy expression, it means

\begin{itemize}
  \item $c p_{\text{new}} \le d_{\text{new}}$
  \item $s' = s + d_{\text{new}} \le q'$
\end{itemize}

The former inequation means that
the deposit of the new bid must be greater than or equal to minimum deposit amount,
that is defined by multiplying $c$, the minimum deposit rate, to $p_{n+1}'$, the bid price .
The latter inequation means that $s'$, the sum of deposits, must be lower than or equal to $q'$, the average of bid prices.
To increase the deposit amount of the new bid
to increase the probability of getting the chance of settlement when the liquidation of collateral occurs,
the new bidder has no way without increasing $p_{\text{new}}$ to increase $q'$.

When the NFT lister newly borrow assets, $a_i$ must follow the constraint

$$
  a_i \le d_i
$$

Trivially the follwing inequation will be satisfied

$$
  b \le s
$$

Deposited amount must be consumed (used for lending resource) in ascending order of interest rates,
so the following constraint must be satisfied.

$$
  a_{i_r(j+1)} = 0 \ \text{if} \ a_{i_r(j)} < d_{i_r(j)}
$$

When the $i_r(j)$ th bid expire, $x_{i_r(j)} = t$, there are two paths to diverge,

First, if the lister configured the automatic refunding when the listing, $a_{i_r(j)}$ the borrowed amount from $i_r(j)$ th bid will be automatically repaid and then
the lister will automatically borrow the same amount from the deposit whose interest rate is the next lowest.

\begin{align}
  a_{i_r(j + k)} &= d_{i_r(j + k)} & \forall k \in \{1 : m-1\} \\
  a_{i_r(j + m)} &= \sum_{l=1}^m \left( d_{i_r(j+l)} - a_{i_r(j+l)} \right) - (1 + r_{i_r(j)}) a_{i_r{j}} &
\end{align}

where $m$ is defined as

$$
  m = \left\{\begin{aligned}
    \min && \ m & \in \mathbb{N} \\
    \text{subject to} && \ (1 + r_{i_r(j)}) a_{i_r{j}} & \le \sum_{l=1}^m {d_{i_r(j+l)} - a_{i_r(j+l)}}
  \end{aligned}\right.
$$


\end{document}
