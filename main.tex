\documentclass[dvipdfmx]{jsarticle}

\usepackage{amsfonts}
\usepackage{amsmath, amssymb, bm}
\usepackage{natbib}
\usepackage{graphicx}
\usepackage{url}
\usepackage{breqn}
\usepackage{tikz}
\usepackage{here}

\newcommand{\argmax}{\mathop{\rm arg~max}\limits}
\newcommand{\argmin}{\mathop{\rm arg~min}\limits}

\title{UnUniFi}

\author{KIMURA Yu, SHIMOJIMA Takeru}

\begin{document}
\numberwithin{equation}{section}

\maketitle

\section{NFT collateral lending}

\subsection{Collateral Liquidation Deposit Auction}

\subsubsection{Definitions}

\begin{itemize}
  \item[$i \in I$] index of bids
  \item[$n = |I|$] number of bids
  \item[$\{d_i\}_{i \in I}$] the deposit amounts of $i$ th bid
  \item[$\{r_i\}_{i \in I}$] the interest rates of $i$ th bid. To unannualize this, an expression $\text{unannualize(r; \text{start}, \text{end})}$ will be used.
  \item[$\{x_i\}_{i \in I}$] the expiration dates of $i$ th bid
  \item[$t$] current time 
  \item[$q$] the average of $p_i$ and the upper bound of $s$
  \item[$s_d$] the sum of deposits and the amount which lister can borrow with NFT as collateral
  \item[$j \in J_i$] index of borrowing from $i$ th bid deposit 
  \item[$\{b_{i,j}\}_{i \in I, j \in J_i}$] means the amount of $j$ th borrowing borrowed from $i$ th bid deposit
  \item[$\{b_i\}_{i \in I}$] means the amount borrowed from $i$ th bid deposit
  \item[$s_b$] the sum of $b_i$
  \item[$\{t_{i,j}\}_{i \in I}$] the start dates of $j$ th borrowing from $i$ th bid deposit
  \item[$i_p(j)$] the index of the $j$ th highest price bid
  \item[$i_d(j)$] the index of the $j$ th highest deposit amount bid
  \item[$i_r(j)$] the index of the $j$ th lowest interest rate bid
  \item[$i_t(j)$] the index of the $j$ th farthest expiration date bid
  \item[$c$] minimum deposit rate
\end{itemize}

$$
  q = \frac{1}{n} \sum_{i \in I} p_i
$$

$$
  s_d = \sum_{i \in I} d_i
$$

$$
  b_i = \sum_{j \in J_i} b_{i,j}
$$

$$
  s_b = \sum_{i \in I} b_i
$$

\subsubsection{Acceptance of the bid}

The lister can accept the bid if the favorable bid price is found.

If this process is executed, $i_p(1)$ th bidder will have the right to settle the deal.
To settle the deal, $i_p(1)$ th bidder have to pay

$$
  p_{i_p(1)} - d_{i_p(1)}
$$

If $i_p(1)$ th bidder do so, he or she will have the posession of the listed NFT.

The lister will receive

$$
  p_{i_p(1)} - \sum_{i \in I} \left( d_i + \sum_{j \in J_i} \text{unannualize}(r_i; t_{i,j}, t) b_{i,j} \right)
$$

Protocol fees are abbreviated.

Each $i \in I$ th bidders will receive

$$
  d_i + \sum_{j \in J_i} \text{unannualize}(r_i; t_{ij}, t) b_{i,j}
$$

If $i_p(1)$ th bidder don't do so, his or her deposit $d_{i_p(1)}$ will be forfeited.

\subsubsection{State transition}

When $(p_{\text{new}}, d_{\text{new}}, r_{\text{new}}, t_{\text{new}})$ will be added to the set of bids, the new bids sequence will be

$$
\begin{aligned}
  I' &= I \cup \{n+1\} \\
  n' &= n + 1 \\
  \{p_i'\}_{i \in I'} &= \{p_i\}_{i \in I} \cup \{p_{\text{new}}\} \\
  \{d_i'\}_{i \in I'} &= \{d_i\}_{i \in I} \cup \{d_{\text{new}}\} \\
  \{r_i'\}_{i \in I'} &= \{r_i\}_{i \in I} \cup \{r_{\text{new}}\} \\
  \{x_i'\}_{i \in I'} &= \{x_i\}_{i \in I} \cup \{x_{\text{new}}\} \\
  q' &= \frac{1}{n'} \sum_{i \in I'} p_i' \\
  s_d' &= \sum_{i \in I'} d_i'
\end{aligned}
$$

where the prime means the next state.

The constraint of $d_{\text{new}}$ is

$$
  c p_{\text{new}} \le d_{\text{new}} \le \min \left\{ p_{\text{new}}, q' - s_d \right\} 
$$

In easy expression, it means

$$
\begin{aligned}
  c p_{\text{new}} &\le d_{\text{new}} \\
  d_{\text{new}} &\le p_{\text{new}} \\
  s_d' = s_d + d_{\text{new}} &\le q'
\end{aligned}
$$

The first inequation means that
the deposit of the new bid must be greater than or equal to minimum deposit amount,
that is defined by multiplying $c$, the minimum deposit rate, to $p_{n+1}'$, the bid price.
The second inequation guarantee that $p_i - d_i \ge 0$.
The third inequation means that $s'$, the sum of deposits, must be lower than or equal to $q'$, the average of bid prices.
To increase the deposit amount of the new bid
to increase the probability of getting the chance of settlement when the liquidation of collateral occurs,
the new bidder has no way without increasing $p_{\text{new}}$ to increase $q'$.

When the NFT lister newly borrow assets, $a_i$ must follow the constraint

$$
  b_i \le d_i
$$

Trivially the follwing inequation will be satisfied

$$
  s_b \le s_d
$$

Deposited amount must be consumed (used for lending resource) in ascending order of interest rates,
so the following constraint must be satisfied.

$$
  b_{i_r(j+1)} = 0 \ \text{if} \ b_{i_r(j)} < d_{i_r(j)}
$$

When the $i_r(k)$ th bid expire, $x_{i_r(k)} = t$, there are two paths to diverge,

First, if the lister configured the automatic refunding when the listing, $a_{i_r(k)}$ the borrowed amount from $i_r(k)$ th bid will be automatically repaid and then
the lister will automatically borrow the same amount from the deposit whose interest rate is the next lowest.

$$
\begin{aligned}
  b_{i_r(l)} &= d_{i_r(l)} & \forall l \in \{1 : m-1\} \\
  b_{i_r(m)} &= \sum_{l=1}^m \left( d_{i_r(l)} - b_{i_r(l)} \right) - \sum_{j \in J_{i_r(k)}} (1 + \text{unannualize}(r_{i_r(k)}; t_{i_r(k),j}, x_{i_r(k)})) b_{i_r(k),j} &
\end{aligned}
$$

where $m$ is defined as

$$
  m = \left\{\begin{aligned}
    \min && \ m & \in \mathbb{N} \\
    \text{subject to} && \ \sum_{j \in J_{i_r(k)}} (1 + \text{unannualize}(r_{i_r(k)}; t_{i_r(k),j}, x_{i_r(k)})) b_{i_r(k),j} & \le \sum_{l=1}^m {d_{i_r(l)} - b_{i_r(l)}}
  \end{aligned}\right.
$$

If the sum of the deposit of remaining bids are insufficient to cover the amount to refund, this process won't be executed and the liquidation process will be executed.

Second, if the lister didn't configure the automatic refunding, the liquidation process will be executed.

\subsubsection{Liquidation process}

for $k \in \{1:n\}$

if $p_{i_d(k)} < s$, continue.

$i_d(k)$ th bidder will have the right of settlement of the deal.
To settle the deal, $i_d(k)$ th bidder have to pay

$$
  p_{i_d(k)} - d_{i_d(k)}
$$

If he or she do so, he or she will have the posession of the listed NFT. break.

If he or she don't do so, his or her deposit $d_{i_d(k)}$ will be forfeited. continue.

Even if all bidders don't exercise the right of settlement of the deal,
the protocol will totally forfeit 

$$
  \sum_{k \in \{1:n\}} d_{i_d(k)} = \sum_{i \in I} d_i = s_d
$$

So there is no possibility of the protocol getting any losses caused by the liquidation process.

If $i_d(k^*)$ th bidder settle the deal,

$$
  p_{i_d(k^*)} + \sum_{l=1}^{k^*} d_{i_d(l)}
$$

will be temporally go to the protocol.

If $\sum_{l=k^*}^n \left(d_{i_d(l)} + \sum_{j \in J_{i_d(l)}} \text{unannualize}(r_{i_d(l)}; t_{i_d(l),j}, t) b_{i_d(l),j} \right) \le p_{i_d(k^*)} + \sum_{l=1}^{k^*} d_{i_d(l)}$

each $i \in \{i_d(l) \mid l \in \{k:n\}\}$ th bidder will receive

$$
  d_i + \sum_{j \in J_i} \text{unannualize}(r_i, t_{i,j}, t) b_{i,j}
$$

and protocol will reveive

$$
  p_{i_d(k^*)} + \sum_{l=1}^{k^*} d_{i_d(l)} - \sum_{l=k^*}^n \left(d_{i_d(l)} + \sum_{j \in J_{i_d(l)}} \text{unannualize}(r_{i_d(l)}; t_{i_d(l),j}, t) b_{i_d(l),j} \right)
$$

else if $\sum_{l=k^*}^n \left(d_{i_d(l)} + \sum_{j \in J_{i_d(l)}} \text{unannualize}(r_{i_d(l)}; t_{i_d(l),j}, t) b_{i_d(l),j} \right) > p_{i_d(k^*)} + \sum_{l=1}^{k^*} d_{i_d(l)}$

each $i \in \{i_d(l) \mid l \in \{k^*:n\}\}$ th bidder will receive

$$
  \frac{d_i + \sum_{j \in J_i} \text{unannualize}(r_i; t_{i,j}, t) b_{i,j}}{\sum_{l=k^*}^n \left(d_{i_d(l)} + \sum_{j \in J_{i_d(l)}} \text{unannualize}(r_{i_d(l)}; t_{i_d(l),j}, t) b_{i_d(l),j} \right)} \left( p_{i_d(k^*)} + \sum_{l=1}^{k^*} d_{i_d(l)} \right)
$$

\section{Interchain Yield Aggregator}

\section{Derivatives}

\subsection{Counterpart Liquidity Pool}

\subsubsection{Definitions}

\begin{itemize}
  \item[$I$] the set of the index of assets in the pool
  \item[$\{p_i\}_{i \in I}$] the price of $i$ th asset in the pool, retrieved from oracle
  \item[$\{l_i\}_{i \in I}$] the amount of the liquidity of $i$ th asset in the pool
  \item[$\{w_i^*\}_{i \in I}$] the target weight of the liquidity of $i$ th asset in the pool
  \item[$\{l_i^*\}_{i \in I}$] the target amount of the liquidity of $i$ th asset in the pool
  \item[$p$] the price of the liquidity provider token
  \item[$s$] the supply of the liquidity provider token
  \item[$\Delta$] the set of derivatives 
  \item[$\Pi_\delta$] the set of positions of a derivative $\delta \in \Delta$
  \item[$\chi(\pi)$] the counterpart position of the position $\pi$
  \item[$r(\pi)$] the imaginary funding rate of the position $\pi$
  \item[$s(\pi)$] the total size of the position $\pi$
\end{itemize}

$l_i^*$ can be calculated with the given values $w_i^*, s, p, \{p_i\}_{i=1}^n$ by $w_i^* = \frac{l_i^* p_i}{s p}$ as

$$
  l_i^* = \frac{w_i^* s p}{p_i}
$$

The price of the liquidity provider token can be calculated as:

$$
  p = \left\{
    \begin{aligned}
      \frac{1}{s} \sum_{i \in I} l_i p_i & \ \text{if} \ s > 0 \\
      \sum_{i \in I} w_i^* p_i & \ \text{if} \ s = 0
    \end{aligned}
  \right.
$$

The minting and redemption of the liquidity provider token is executed with this price $p$.

\subsubsection{Imaginary funding rate}

In Central Limit Order Book (CLOB) model, index prices will be retrieved via oracles and mark prices will be determined in CLOB.
Funding rates plays a role of making the difference between the mark price and the index price close to zero.

On the other hand, in Counterpart Liquidity Pool model, mark prices are equivalent to index prices. Both of them are retrieved via oracles.
Counterpart Liquidity Pool will take the counterpart position of a trader's order, so there is no time to wait for matchmaking.
However, in this model, if traders get profits, the pool and at the same time liquidity providers get losses.
To tackle this problem, imaginary funding rate exists.
If the net position of traders lean to one side, the imaginary funding rate work to make the net position of traders neutral.
The neutral net position of traders means the neutral position of the pool an at the same time liquidity providers.
In the perspective of economics, it can be expressed that this model unifies the conventional funding rate and the time cost of waiting for matchmaking to the imaginary funding rate.

\subsubsection{Perpetual futures}

$\text{PF} \in \Delta$ serves perpetual futures.

Positions are

$$
  \Pi_{\text{PF}} = \{\text{long}(i, p), \text{short}(i, p)\}
$$

where $i$ is the index of assets and $p$ is the position price.

The relations of counterpart position are

$$
\begin{aligned}
  \chi(\text{long}(i)) & = \text{short}(i) \\
  \chi(\text{short}(i)) & = \text{long}(i)
\end{aligned}
$$

$$
  \forall p \in \mathbb{R}_+, \ r(\text{long}(i, p)) \propto \int_{\mathbb{R}_+} (s(\text{long}(i, q)) - s(\text{short}(i, q))) dq
$$

Notice that $r(\pi) = -r(\chi(\pi))$.

\subsubsection{Perpetual options}

$\text{PO} \in \Delta$ serves perpetual options.

Positions are

$$
  \Pi_{\text{PO}} = \{\text{long\_call}(i, p), \text{long\_put}(i, p), \text{short\_call}(i, p), \text{short\_put}(i, p)\}
$$

where $i$ is the index of assets and $p$ is the strike price.

The relations of counterpart position are

$$
\begin{aligned}
  \chi(\text{long\_call}(i, p)) & = \text{short\_call}(i, p) \\
  \chi(\text{short\_call}(i, p)) & = \text{long\_call}(i, p) \\
  \chi(\text{long\_put}(i, p)) & = \text{short\_put}(i, p) \\
  \chi(\text{short\_put}(i, p)) & = \text{long\_put}(i, p)
\end{aligned}
$$

\section{Ecosystem Incentive}

\subsection{Decentralized Frontend}

\end{document}
