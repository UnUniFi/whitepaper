\documentclass[dvipdfmx]{jsarticle}

\usepackage{amsfonts}
\usepackage{amsmath, amssymb, bm}
\usepackage[english]{babel}
\usepackage{natbib}
\usepackage{graphicx}
\usepackage{url}
\usepackage{breqn}
\usepackage{tikz}
\usepackage{here}

\newcommand{\argmax}{\mathop{\rm arg~max}\limits}
\newcommand{\argmin}{\mathop{\rm arg~min}\limits}

\title{UnUniFi}

\author{KIMURA Yu, SHIMOJIMA Takeru}

\begin{document}
\numberwithin{equation}{section}

\maketitle

\begin{abstract}
In the rapidly evolving landscape of blockchain technology, the management of on-chain assets continues to expand in complexity and capability, giving rise to the urgent need for innovative management solutions. The current methods in which these assets can be deployed to competitively earn yield are ripe for disruption. UnUniFi‘s dynamic Asset Management Platform aims to revolutionize the efficient operation of these assets across many blockchains. Cross-chain support, access to the best-in-class yield opportunities, and permissionless vault creation are just a few of the exciting features available on UnUniFi (pronounced “un-unify”). Supported by esteemed investors like Gumi Cryptos Capital and more, UnUniFi is on a mission to democratize asset management, catering to both retail and institutional stakeholders. Let’s take a deep dive into what UnUniFi is all about.
\end{abstract}

\section{Problems}

\subsection{Problems in DeFi asset management for Users}
Users must spend a large amount of time and effort across many apps and blockchains to find efficient DeFi strategies. Users cannot easily access complex strategies such as CeFi, RWAs, or Collateralized Assets as part of their portfolio management. Users cannot easily customize their exposure to specific DeFi + CeFi strategies.

\subsection{Problems in liquidity management for DeFi Protocols and Institutions}
Strategy providers (yield pool creators/borrowers) suffer from fragmented liquidity across chains due to the difficulties and high cost of creating multi-chain support.

\section{UnUniFi: A New Standard for Cross-Chain Asset Management}
UnUniFi distinguishes itself from other Asset Management and Yield Aggregation platforms by addressing the limitations and challenges faced by existing solutions. UnUniFi leverages the Cosmos SDK and IBC to ensure maximum scalability; speed; security; and cross-chain interoperability, making it an ideal hub for Asset Management. Regardless of a user’s source chain, UnUniFi streamlines the process, ensuring that every DeFi user can seamlessly access the platform. UnUniFi empowers users to access an unlimited number of yield strategies, while sourcing strategies from the best DeFi protocols and CeFi services. This unique DeFi + CeFi approach enables users to receive unprecedented customizability when managing their assets, all within one easy-to-use cross-chain dashboard.
Meanwhile, strategy providers can leverage UnUniFi’s cross-chain support to deploy their contracts and source global liquidity across blockchains. Professionally operated and custodied strategies are compliant for Institutional and Accredited Investors. UnUniFi’s asset management platform is a one-stop solution enabling individuals, DAOs, funds, and other protocols to earn yield against unlimited strategies, featuring the best DeFi strategies, CeFi strategies, RWAs, and Collateralized Assets.


\section{UnUniFi’s Core Application: UnUniFi Yield Aggregator (UYA)}

Powering the Asset Management Platform is UnUniFi’s core application, the UnUniFi Yield Aggregator, or UYA. The UYA app functions as a cross-chain hub, enabling users to explore the true possibilities of cross-chain asset operation. The unlimited creation of Yield Strategies and Vaults fosters competitive and diverse opportunities.

\subsection{How are Yield Strategies aggregated (sourced)}
\begin{itemize}
  \item Chain-agnostic, seamless integration with the best DeFi protocols + CeFi Services, enabling access to the higher performing Yield Strategies. In this method, UnUniFi will directly integrate with the source Pool, even if it is on an external chain.
  \item Unlimited strategy creation, allowing any strategy provider to deploy their pool on UnUniFi while fostering cross-chain interoperability and minimizing liquidity fragmentation.
\end{itemize}


\subsection{What are the types of Yield Strategies supported?}
\begin{enumerate}
  \item DeFi Strategies(e.g. DeFi Protocols, AMM)
  \item CeFi Strategies(KYC required)(e.g. centralized lending (CeFi RWA))
  \item Collateralized Assets(e.g. non-fungible assets, Fine-Art, Luxury, Tokenized Assets, Tokenized DeFi Positions)
\end{enumerate}

\subsection{How do users use the UYA?}
\begin{itemize}
  \item Basic Users can manage (deposit) any asset type into a vault of their choice, with the vault being composed of one or more underlying strategies. For the average user, the UYA App will be able to display the highest performing yield strategies and vaults, easily allowing them to select a pre-created vault and automatically participate in the highest performing strategies.
  \item Advanced Users can leverage an unprecedented level of customization, with the UYA App allowing anyone to autonomously manage their portfolio exposure by creating their own Vault (permissionless vault creation). This allows for the combination and selection of DeFi + CeFi strategies within the same Vault, while allowing the creator to also specify the weightings of the underlying strategies. Anyone can create a vault, so long as it has not already been created (duplicated vaults will be rejected).
  \item Institutional Users and Accredited Investors (e.g. Funds as Lenders) can manage their assets by depositing funds into CeFi + DeFi strategies. Professionally managed and compliant CeFi + DeFi strategies will be supported via Qualified Custodian.
\end{itemize}

\subsection{The flow of UYA}
\begin{enumerate}
  \item Deposit tokens into the IYA module for UnUniFi Protocol.
  \item Specify a strategy based on the information provided by the data provider and select contract
  \item Initiate participation by sending TX from the contract to the other chain through an interchain account.
\end{enumerate}

\subsection{The process of receiving rewards}
\begin{enumerate}
  \item Users claim rewards in the UYA module for UnUniFi Protocol.
  \item The UYA receives rewards on the contract the user selected.
  \item The contract receives from the chain from which the user had initiated the participation.
  \item The contract sends rewards from the other chain to UnUniFi Protocol.
  \item UYA sends tokens to the user.
\end{enumerate}


\subsection{How do Strategy Providers use the UYA?}
Strategy providers can create and deploy their contracts/pools on UnUniFi. This mechanism is cross-chain enabled and can be utilized by any party. Below are two examples:
\begin{itemize}
  \item DeFi Protocols (e.g. Automated Market Makers) can add their contracts to the UYA to source global liquidity, fostering cross-chain interoperability and minimizing liquidity fragmentation (by eliminating the need to deploy a pool on multiple chains).
  \item Institutions (e.g. Funds as Borrowers) can create their own lending pool for users, allowing them to aggregate global liquidity and borrow funds from the pool.
\end{itemize}


\section{NFT collateral lending}

\subsection{Collateral Liquidation Deposit Auction}

\subsubsection{Definitions}

\begin{itemize}
  \item[$i \in I$] index of bids
  \item[$n = |I|$] number of bids
  \item[$\{d_i\}_{i \in I}$] the deposit amounts of $i$ th bid
  \item[$\{r_i\}_{i \in I}$] the interest rates of $i$ th bid. To unannualize this, an expression $\text{unannualize(r; \text{start}, \text{end})}$ will be used.
  \item[$\{x_i\}_{i \in I}$] the expiration dates of $i$ th bid
  \item[$t$] current time 
  \item[$q$] the average of $p_i$ and the upper bound of $s$
  \item[$s_d$] the sum of deposits and the amount which lister can borrow with NFT as collateral
  \item[$j \in J_i$] index of borrowing from $i$ th bid deposit 
  \item[$\{b_{i,j}\}_{i \in I, j \in J_i}$] means the amount of $j$ th borrowing borrowed from $i$ th bid deposit
  \item[$\{b_i\}_{i \in I}$] means the amount borrowed from $i$ th bid deposit
  \item[$s_b$] the sum of $b_i$
  \item[$\{t_{i,j}\}_{i \in I}$] the start dates of $j$ th borrowing from $i$ th bid deposit
  \item[$i_p(j)$] the index of the $j$ th highest price bid
  \item[$i_d(j)$] the index of the $j$ th highest deposit amount bid
  \item[$i_r(j)$] the index of the $j$ th lowest interest rate bid
  \item[$i_t(j)$] the index of the $j$ th farthest expiration date bid
  \item[$c$] minimum deposit rate
\end{itemize}

$$
  q = \frac{1}{n} \sum_{i \in I} p_i
$$

$$
  s_d = \sum_{i \in I} d_i
$$

$$
  b_i = \sum_{j \in J_i} b_{i,j}
$$

$$
  s_b = \sum_{i \in I} b_i
$$

\subsubsection{Acceptance of the bid}

The lister can accept the bid if the favorable bid price is found.

If this process is executed, $i_p(1)$ th bidder will have the right to settle the trade.
To settle the trade, $i_p(1)$ th bidder have to pay

$$
  p_{i_p(1)} - d_{i_p(1)}
$$

If $i_p(1)$ th bidder proceeds, he or she will obtain posession of the listed NFT.

The lister will receive

$$
  p_{i_p(1)} - \sum_{i \in I} \left( d_i + \sum_{j \in J_i} \text{unannualize}(r_i; t_{i,j}, t) b_{i,j} \right)
$$

Protocol fees are abbreviated.

Each $i \in I$ th bidders will receive

$$
  d_i + \sum_{j \in J_i} \text{unannualize}(r_i; t_{ij}, t) b_{i,j}
$$

If $i_p(1)$ th bidder decide not to proceed, his or her deposit $d_{i_p(1)}$ will be forfeited.

\subsubsection{State transition}

When $(p_{\text{new}}, d_{\text{new}}, r_{\text{new}}, t_{\text{new}})$ will be added to the set of bids, the new bids sequence will be

$$
\begin{aligned}
  I' &= I \cup \{n+1\} \\
  n' &= n + 1 \\
  \{p_i'\}_{i \in I'} &= \{p_i\}_{i \in I} \cup \{p_{\text{new}}\} \\
  \{d_i'\}_{i \in I'} &= \{d_i\}_{i \in I} \cup \{d_{\text{new}}\} \\
  \{r_i'\}_{i \in I'} &= \{r_i\}_{i \in I} \cup \{r_{\text{new}}\} \\
  \{x_i'\}_{i \in I'} &= \{x_i\}_{i \in I} \cup \{x_{\text{new}}\} \\
  q' &= \frac{1}{n'} \sum_{i \in I'} p_i' \\
  s_d' &= \sum_{i \in I'} d_i'
\end{aligned}
$$

where the prime means the next state.

The constraint of $d_{\text{new}}$ is

$$
  c p_{\text{new}} \le d_{\text{new}} \le \min \left\{ p_{\text{new}}, q' - s_d \right\} 
$$

In easy expression, it means

$$
\begin{aligned}
  c p_{\text{new}} &\le d_{\text{new}} \\
  d_{\text{new}} &\le p_{\text{new}} \\
  s_d' = s_d + d_{\text{new}} &\le q'
\end{aligned}
$$

The first inequation means that
the deposit of the new bid must be greater than or equal to minimum deposit amount,
that is defined by multiplying $c$, the minimum deposit rate, to $p_{n+1}'$, the bid price.
The second inequation guarantee that $p_i - d_i \ge 0$.
The third inequation means that $s'$, the sum of deposits, must be lower than or equal to $q'$, the average of bid prices.
To increase the deposit amount of the new bid
and increase the probability of getting settlement when the liquidation of collateral occurs,
the new bidder must increase $p_{\text{new}}$ to increase $q'$.

When the NFT lister newly borrow assets, $a_i$ must follow the constraint

$$
  b_i \le d_i
$$

Additionally the follwing inequation will be satisfied

$$
  s_b \le s_d
$$

Deposited amount will therefore be consumed (used for lending resource) in ascending order of interest rates,
so the following constraint must be satisfied.

$$
  b_{i_r(j+1)} = 0 \ \text{if} \ b_{i_r(j)} < d_{i_r(j)}
$$

When the $i_r(k)$ th bid expire, $x_{i_r(k)} = t$, there are two distinct paths to select,

First, if the lister configured the automatic refunding when initiating the listing, $a_{i_r(k)}$ the borrowed amount from $i_r(k)$ th bid will be automatically repaid.
The lister then will automatically borrow the same amount from the deposit whose interest rate is the next lowest.

$$
\begin{aligned}
  b_{i_r(l)} &= d_{i_r(l)} & \forall l \in \{1 : m-1\} \\
  b_{i_r(m)} &= \sum_{l=1}^m \left( d_{i_r(l)} - b_{i_r(l)} \right) - \sum_{j \in J_{i_r(k)}} (1 + \text{unannualize}(r_{i_r(k)}; t_{i_r(k),j}, x_{i_r(k)})) b_{i_r(k),j} &
\end{aligned}
$$

where $m$ is defined as

$$
  m = \left\{\begin{aligned}
    \min && \ m & \in \mathbb{N} \\
    \text{subject to} && \ \sum_{j \in J_{i_r(k)}} (1 + \text{unannualize}(r_{i_r(k)}; t_{i_r(k),j}, x_{i_r(k)})) b_{i_r(k),j} & \le \sum_{l=1}^m {d_{i_r(l)} - b_{i_r(l)}}
  \end{aligned}\right.
$$

If the sum of the deposits of remaining bids are insufficient to cover the refund amount, this process cannot execute and instead liquidation process will be executed.

Second, if the lister didn't configure the automatic refunding, the liquidation process will be executed.

\subsubsection{Liquidation process}

for $k \in \{1:n\}$

if $p_{i_d(k)} < s$, continue.

$i_d(k)$ th bidder will have the right of settlement of the deal.
To settle the deal, $i_d(k)$ th bidder have to pay

$$
  p_{i_d(k)} - d_{i_d(k)}
$$

If they do so, then they will have the posession of the listed NFT.

If they don't, then their deposit $d_{i_d(k)}$ will be forfeited.

Even if all bidders decide not to exercise their settlement right for the deal,
the protocol will completely forfeit 

$$
  \sum_{k \in \{1:n\}} d_{i_d(k)} = \sum_{i \in I} d_i = s_d
$$

Therefore, there is no possibility of the protocol accruing any losses caused by the liquidation process.

If $i_d(k^*)$ th bidder settles the deal,

$$
  p_{i_d(k^*)} + \sum_{l=1}^{k^*} d_{i_d(l)}
$$

will be temporally sent to the protocol.

If $\sum_{l=k^*}^n \left(d_{i_d(l)} + \sum_{j \in J_{i_d(l)}} \text{unannualize}(r_{i_d(l)}; t_{i_d(l),j}, t) b_{i_d(l),j} \right) \le p_{i_d(k^*)} + \sum_{l=1}^{k^*} d_{i_d(l)}$

each $i \in \{i_d(l) \mid l \in \{k:n\}\}$ th bidder will receive

$$
  d_i + \sum_{j \in J_i} \text{unannualize}(r_i, t_{i,j}, t) b_{i,j}
$$

and protocol will reveive

$$
  p_{i_d(k^*)} + \sum_{l=1}^{k^*} d_{i_d(l)} - \sum_{l=k^*}^n \left(d_{i_d(l)} + \sum_{j \in J_{i_d(l)}} \text{unannualize}(r_{i_d(l)}; t_{i_d(l),j}, t) b_{i_d(l),j} \right)
$$

else if $\sum_{l=k^*}^n \left(d_{i_d(l)} + \sum_{j \in J_{i_d(l)}} \text{unannualize}(r_{i_d(l)}; t_{i_d(l),j}, t) b_{i_d(l),j} \right) > p_{i_d(k^*)} + \sum_{l=1}^{k^*} d_{i_d(l)}$

each $i \in \{i_d(l) \mid l \in \{k^*:n\}\}$ th bidder will receive

$$
  \frac{d_i + \sum_{j \in J_i} \text{unannualize}(r_i; t_{i,j}, t) b_{i,j}}{\sum_{l=k^*}^n \left(d_{i_d(l)} + \sum_{j \in J_{i_d(l)}} \text{unannualize}(r_{i_d(l)}; t_{i_d(l),j}, t) b_{i_d(l),j} \right)} \left( p_{i_d(k^*)} + \sum_{l=1}^{k^*} d_{i_d(l)} \right)
$$

\section{Derivatives}

\subsection{Counterpart Liquidity Pool}

\subsubsection{Definitions}

\begin{itemize}
  \item[$I$] the set of the index of assets in the pool
  \item[$\{p_i\}_{i \in I}$] the price of $i$ th asset in the pool, retrieved from oracle
  \item[$\{l_i\}_{i \in I}$] the amount of the liquidity of $i$ th asset in the pool
  \item[$\{w_i^*\}_{i \in I}$] the target weight of the liquidity of $i$ th asset in the pool
  \item[$\{l_i^*\}_{i \in I}$] the target amount of the liquidity of $i$ th asset in the pool
  \item[$p$] the price of the liquidity provider token
  \item[$s$] the supply of the liquidity provider token
  \item[$\Delta$] the set of derivatives 
  \item[$\Pi_\delta$] the set of positions of a derivative $\delta \in \Delta$
  \item[$\chi(\pi)$] the counterpart position of the position $\pi$
  \item[$r(\pi)$] the imaginary funding rate of the position $\pi$
  \item[$s(\pi)$] the total size of the position $\pi$
\end{itemize}

$l_i^*$ can be calculated with the given values $w_i^*, s, p, \{p_i\}_{i=1}^n$ by $w_i^* = \frac{l_i^* p_i}{s p}$ as

$$
  l_i^* = \frac{w_i^* s p}{p_i}
$$

The price of the liquidity provider token can be calculated as:

$$
  p = \left\{
    \begin{aligned}
      \frac{1}{s} \sum_{i \in I} l_i p_i & \ \text{if} \ s > 0 \\
      \sum_{i \in I} w_i^* p_i & \ \text{if} \ s = 0
    \end{aligned}
  \right.
$$

The minting and redemption of the liquidity provider token is executed with this price $p$.

\subsubsection{Imaginary funding rate}

In a Central Limit Order Book (CLOB) model, index prices are retrieved via oracles and mark prices are determined in the CLOB. 
Funding rates ensure the differences between the mark price and the index price is close to zero.

On the other hand, in the Counterpart Liquidity Pool model, mark prices are equivalent to index prices, and both are retrieved via oracles. 
The Counterpart Liquidity Pool will take the counterpart position of a trader’s order, so there is no time required for matchmaking. 
However, in this model, if traders acquire profits, the pool and liquidity providers simultaneously incur losses. 
To tackle this problem, an imaginary funding rate exists. 
If the net position of traders leans to one side, the imaginary funding rate works to make the net position of traders neutral.  
The neutral net position of traders means a neutral position for both the pool and the liquidity providers. 
In the perspective of economics, it can be expressed that this model unifies the conventional funding rate and the time cost of waiting for matchmaking to the imaginary funding rate.

\subsubsection{Perpetual futures}

$\text{PF} \in \Delta$ serves perpetual futures.

Positions are

$$
  \Pi_{\text{PF}} = \{\text{long}(i, p), \text{short}(i, p)\}
$$

where $i$ is the index of assets and $p$ is the position price.

The relations of counterpart position are

$$
\begin{aligned}
  \chi(\text{long}(i)) & = \text{short}(i) \\
  \chi(\text{short}(i)) & = \text{long}(i)
\end{aligned}
$$

$$
  \forall p \in \mathbb{R}_+, \ r(\text{long}(i, p)) \propto \int_{\mathbb{R}_+} (s(\text{long}(i, q)) - s(\text{short}(i, q))) dq
$$

Notice that $r(\pi) = -r(\chi(\pi))$.

\subsubsection{Perpetual options}

$\text{PO} \in \Delta$ serves perpetual options.

Positions are

$$
  \Pi_{\text{PO}} = \{\text{call\_long}(i, p), \text{call\_short}(i, p), \text{put\_long}(i, p), \text{put\_short}(i, p)\}
$$

where $i$ is the index of assets and $p$ is the strike price.

The relations of counterpart position are

$$
\begin{aligned}
  \chi(\text{call\_long}(i, p)) & = \text{call\_short}(i, p) \\
  \chi(\text{call\_short}(i, p)) & = \text{call\_long}(i, p) \\
  \chi(\text{put\_long}(i, p)) & = \text{put\_short}(i, p) \\
  \chi(\text{put\_short}(i, p)) & = \text{put\_long}(i, p)
\end{aligned}
$$

\section{Ecosystem Incentive}
We recently analyzed the state of various Crypto and DeFi ecosystems and their applications, and asked the question, are dApps truly decentralized? 
If the viability of a DeFi protocol is dependent on their frontend, how can their dApp claim to satisfy decentralization concerns? 
Our research indicated that the current state of dApps is that many are not truly decentralized and users must often rely on a single frontend deployment. 
In this scenario, reliance on a single frontend can create a single point of failure, and overall weakness in the reliability of a protocol.

\subsection{Single Point of Failure}
Generally, users rely on the established Frontend to interact with a specific dApp or Contract. 
In a real-world example, if a developer understands how a dApp operates, say Opensea for example, they can call the contract directly without needing to use the Frontend. 
Meanwhile, regular users must rely on the Opensea Frontend in order to interact with its blockchain components. 
If the Frontend for Opensea stopped working, most users would not know how to interact with the contract. 
Having one frontend creates a single point of failure. 
So how do we avoid this? Our solution proposes the ability to create multiple independent frontends, which in turn embodies a true “decentralized” model (eliminating the single point of failure).

\subsection{Our Solution- Truly Decentralized Frontend(s)}
Through our own protocol development, we considered these exact same questions, ultimately allowing us to acknowledge the limitations of current solutions in hopes to create a better decentralized ecosystem.
This reflection directed us to create an ecosystem where our API, client library, Bubble App plugin, and frontend incentive module can be utilized by anyone; using a platform like Bubble allows anyone to create unique applications with their no-code app development tools. 
Through hyper-accessible frontend deployment, we can be the first ecosystem to successfully decentralize our frontend without a single point of failure.

\subsection{Ecosystem Development via Community Inclusion}
Normally, only experienced developers can contribute to ecosystem expansion when building dApps. 
However, with our Decentralized Frontend, we can expand the ecosystem to include non-developers. 
This contribution to community inclusion inherently furthers our ecosystem by allowing anyone to build onto it. 
A Decentralized Frontend will also simultaneously inspire both localization and mass adoption by allowing the creation of a Frontend that is hyper-focused for each region and category of usage.
We believe that the “Decentralized Frontend” is an effective method to expand our ecosystem. 

\subsection{How do we Incentivize Frontend Development?}
It is one thing to create an environment where our frontend can specifically be distributed and decentralized based on user needs, but how do we encourage these boutique integrations or unique Frontends? 
Our ecosystem incentive modules aim to reward early adopters who help to expand and realize our decentralized ecosystem.
Our latest implementation of this incentive model is a system that distributes rewards to a frontend creator based on the NFT trading fee (excluding gas fee) collected when using that specific creator’s frontend. 
Therefore, reward distribution is perfectly proportional to the actual trading volume.
In theory, this module will provide a competitive incentive for the parties which bring value to our ecosystem — motivating Frontend service creators (developers). 
These unique Frontend service creators can then become candidates to receive the Ecosystem Incentive Reward, generated from the various fees accumulated.

\section{GUU token distribution}
The native cryptographically-secure fungible protocol token of the UnUniFi Protocol (ticker symbol GUU) is a transferable representation of attributed governance and utility functions specified in the protocol/code of the UnUniFi Protocol, and which is designed to be used solely as an interoperable utility token thereon.
GUU is a functional multi-utility token which will be used as the medium of exchange between participants on the UnUniFi Protocol in a decentralised manner. 
The goal of introducing GUU is to provide a convenient and secure mode of payment and settlement between participants who interact within the ecosystem on the UnUniFi Protocol without any intermediaries such as centralised third party entity/institution/credit. 
It is not, and not intended to be, a medium of exchange accepted by the public (or a section of the public) as payment for goods or services or for the discharge of a debt; nor is it designed or intended to be used by any person as payment for any goods or services whatsoever that are not exclusively provided by the issuer. 
GUU does not in any way represent any shareholding, ownership, participation, right, title, or interest in the Company, the Distributor, their respective affiliates, or any other company, enterprise or undertaking, nor will GUU entitle token holders to any promise of fees, dividends, revenue, profits or investment returns, and are not intended to constitute securities in Panama, Singapore or any relevant jurisdiction. 
GUU may only be utilised on the UnUniFi Protocol, and ownership of the same carries no rights, express or implied, other than the right to use GUU as a means to enable usage of and interaction within the UnUniFi Protocol. 
The secondary market pricing of GUU is not dependent on the effort of the UnUniFi Project Contributors, and there is no token functionality or scheme designed to control or manipulate such secondary pricing.
Further, GUU provides the economic incentives which will be distributed to encourage users to exert efforts towards contribution and participation in the ecosystem on the UnUniFi Protocol, thereby creating a mutually beneficial system where every participant is fairly compensated for its efforts. 
GUU is an integral and indispensable part of the UnUniFi Protocol, because without GUU, there would be no incentive for users to expend resources to participate in activities or provide services for the benefit of the entire ecosystem on the UnUniFi Protocol. 
Given that additional GUU will be awarded to a user based only on its actual usage, activity and efforts made on the UnUniFi Protocol and/or proportionate to the frequency and volume of transactions, users of the UnUniFi Protocol and/or holders of GUU which did not actively participate will not receive any GUU incentives.
The UnUniFi Protocol itself is simply a blockchain protocol which, as designed, does not own or run any computing/storage servers. 
It relies on an open, decentralised network of validator nodes which operate on an open source algorithm to verify, process and share pricing information and platform transactions. 
Accordingly, third-party computing/bandwidth/storage resources are required for maintaining the network and providers of these services / resources would be paid for the consumption of resources to maintain network integrity (i.e. "mining" on the UnUniFi Protocol), and GUU will be used as the native currency to quantify and pay the costs of the consumed resources.
As an indication of commitment to the system (to filter genuine transactions and minimise network chatter) and to encourage higher performance and availability, validators would be required to place an amount of GUU as security deposit before it may participate in network maintenance/support. 
Conversely, GUU is also used as a deterrent for penalising validators or providers for various offences (e.g. erroneous data or other malicious acts) - penalties include reduction of reputation rating, deducting incentives, deducting the stake (or part thereof) or temporarily or permanently expelling the validator / provider from the pool.
It is anticipated that the community of GUU holders would comprise a diverse field of developers, professionals and supporters of the project to develop the UnUniFi Protocol (including without limitation experts in software development, blockchain technology, cryptography, artificial intelligence, law or finance), which will be able to share and exchange balanced views on the overall direction of the project. 
To promote decentralised community governance for the network, GUU would allow holders to propose and vote on governance proposals to determine future features, upgrades and/or parameters of the UnUniFi Protocol, or provide feedback, with voting weight calculated in proportion to the tokens staked. 
The right to vote is restricted solely to voting on features of the UnUniFi Protocol; it does not entitle GUU holders to vote on the operation and management of the Company, its affiliates, or their assets or the disposition of such assets to token holders, or select the board of directors or similar bodies of these entities, or determine the development direction of these entities, nor does GUU constitute any equity interest in any of these entities or any collective investment scheme; the arrangement is not intended to be any form of joint venture or partnership. 
\clearpage
\begin{table}[htb]
  \begin{tabular}{|l||l|l|}  \hline
    Usage  & Supply[\%]  & Vesting \\ \hline
    Ecosystem Development & 30\% & 
    \begin{tabular}{l}
      Vesting term depending on the situation. \\ Minimum 1-yr linear vesting
    \end{tabular}\\ \hline
    Assignment for validators & 15\% & 1-yr full locked, linear 12 months \\ \hline
    Assignment for UnUniFi Project Contributors & 15\% &
    \begin{tabular}{l}
      1-yr full locked, linear 36 months \\ 1-yr full locked, linear 60 months
      \end{tabular}\\ \hline
    Assignment for UnUniFi Development Fund  & 5\% &
    \begin{tabular}{l}
      1-yr full locked, linear 36 months \\ 1-yr full locked, linear 60 months
      \end{tabular}\\ \hline
    Marketing  & 14\% & Early purchases(1-yr full locked, linear 36 months) \\ \hline
    Treasury  & 10\% &
    \begin{tabular}{l}
      Vesting term depending on the situation. \\ Minimum 1-yr linear vesting
    \end{tabular}\\ \hline
    Advisor  & 1\% & linear 6 months from the time of grant \\ \hline
    Assignment for business partners  & 10\% & 1-yr full locked, linear 24 months \\ \hline
  \end{tabular}
\end{table}

\section{Tokens specifications}

\begin{table}[htb]
  \begin{tabular}{|l|l|l|}  \hline
    Name & Symbol & Denom in blockchain \\ \hline
    GUU & GUU & uguu \\ \hline
  \end{tabular}
\end{table}

\section{Tokens specifications}

\begin{table}[htb]
  \begin{tabular}{|l|l|l|}  \hline
    initial Supply & 1,000,000,000 GUU \\ \hline
    Inflation rate range & 7\% - 20\% \\ \hline
  \end{tabular}
\end{table}

\section{Governance specifications}

\begin{table}[htb]
  \begin{tabular}{|l|l|l|}  \hline
    Minimum deposit for voting & 1,000,000uguu=1GUU \\ \hline
  \end{tabular}
\end{table}

\section{Staking specifications}

\begin{table}[htb]
  \begin{tabular}{|l|l|l|}  \hline
    Max validators & 100 \\ \hline
    Bonding denom & uguu \\ \hline
  \end{tabular}
\end{table}


\section{About Us}
The UnUniFi Protocol is owned by the UnUniFi Project Contributors.

\section{Legal Disclaimer}
PLEASE READ THE ENTIRETY OF THIS "LEGAL DISCLAIMER" SECTION CAREFULLY. NOTHING HEREIN CONSTITUTES LEGAL, FINANCIAL, BUSINESS OR TAX ADVICE AND YOU ARE STRONGLY ADVISED TO CONSULT YOUR OWN LEGAL, FINANCIAL, TAX OR OTHER PROFESSIONAL ADVISOR(S) BEFORE ENGAGING IN ANY ACTIVITY IN CONNECTION HEREWITH. NEITHER UNUNIFI S.A. (THE COMPANY), ANY OF THE PROJECT CONTRIBUTORS (THE UNUNIFI PROJECT CONTRIBUTORS) WHO HAVE WORKED ON THE UNUNIFI PROTOCOL (AS DEFINED HEREIN) OR PROJECT TO DEVELOP THE UNUNIFI PROTOCOL IN ANY WAY WHATSOEVER, ANY DISTRIBUTOR AND/OR VENDOR OF GUU TOKENS (OR SUCH OTHER RE-NAMED OR SUCCESSOR TICKER CODE OR NAME OF SUCH TOKENS) (THE DISTRIBUTOR), NOR ANY SERVICE PROVIDER SHALL BE LIABLE FOR ANY KIND OF DIRECT OR INDIRECT DAMAGE OR LOSS WHATSOEVER WHICH YOU MAY SUFFER IN CONNECTION WITH ACCESSING THE PAPER, DECK OR MATERIAL RELATING TO GUU (THE TOKEN DOCUMENTATION) AVAILABLE ON THE WEBSITE AT HTTPS://UNUNIFI.IO/ (THE WEBSITE, INCLUDING ANY SUB-DOMAINS THEREON) OR ANY OTHER WEBSITES OR MATERIALS PUBLISHED OR COMMUNICATED BY THE COMPANY OR ITS REPRESENTATIVES FROM TIME TO TIME.\\
Project purpose: You agree that you are acquiring GUU to participate in the UnUniFi Protocol and to obtain services on the ecosystem thereon. The Company, the Distributor and their respective affiliates would develop and contribute to the underlying source code for the UnUniFi Protocol. The Company is acting solely as an arms’ length third party in relation to the GUU distribution, and not in the capacity as a financial advisor or fiduciary of any person with regard to the distribution of GUU.\\
Nature of the Token Documentation: The Token Documentation is a conceptual paper that articulates some of the main design principles and ideas for the creation of a digital token to be known as GUU. The Token Documentation and the Website are intended for general informational purposes only and do not constitute a prospectus, an offer document, an offer of securities, a solicitation for investment, any offer to sell any product, item, or asset (whether digital or otherwise), or any offer to engage in business with any external individual or entity provided in said documentation. The information herein may not be exhaustive and does not imply any element of, or solicit in any way, a legally-binding or contractual relationship. There is no assurance as to the accuracy or completeness of such information and no representation, warranty or undertaking is or purported to be provided as to the accuracy or completeness of such information. Where the Token Documentation or the Website includes information that has been obtained from third party sources, the Company, the Distributor, their respective affiliates and/or the UnUniFi Project Contributors have not independently verified the accuracy or completeness of such information. Further, you acknowledge that the project development roadmap, platform/network functionality are subject to change and that the Token Documentation or the Website may become outdated as a result; and neither the Company nor the Distributor is under any obligation to update or correct this document in connection therewith.\\
Validity of Token Documentation and Website: Nothing in the Token Documentation or the Website constitutes any offer by the Company, the Distributor, or the UnUniFi Project Contributors to sell any GUU (as defined herein) nor shall it or any part of it nor the fact of its presentation form the basis of, or be relied upon in connection with, any contract or investment decision. Nothing contained in the Token Documentation or the Website is or may be relied upon as a promise, representation or undertaking as to the future performance of the UnUniFi Protocol. The agreement between the Distributor (or any third party) and you, in relation to any distribution or transfer of GUU, is to be governed only by the separate terms and conditions of such agreement.\\
The information set out in the Token Documentation and the Website is for community discussion only and is not legally binding. No person is bound to enter into any contract or binding legal commitment in relation to the acquisition of GUU, and no digital asset or other form of payment is to be accepted on the basis of the Token Documentation or the Website. The agreement for distribution of GUU and/or continued holding of GUU shall be governed by a separate set of Terms and Conditions or Token Distribution Agreement (as the case may be) setting out the terms of such distribution and/or continued holding of GUU (the Terms and Conditions), which shall be separately provided to you or made available on the Website. The Terms and Conditions must be read together with the Token Documentation. In the event of any inconsistencies between the Terms and Conditions and the Token Documentation or the Website, the Terms and Conditions shall prevail.\\
Deemed Representations and Warranties: By accessing the Token Documentation or the Website (or any part thereof), you shall be deemed to represent and warrant to the Company, the Distributor, their respective affiliates, and the UnUniFi Project Contributors as follows:\\
(a)	in any decision to acquire any GUU, you have not relied and shall not rely on any statement set out in the Token Documentation or the Website;\\
(b)	you shall at your own expense ensure compliance with all laws, regulatory requirements and restrictions applicable to you (as the case may be);\\
(c)	you acknowledge, understand and agree that GUU may have no value, there is no guarantee or representation of value or liquidity for GUU, and GUU is not an investment product nor is it intended for any speculative investment whatsoever;\\
(d)	none of the Company, the Distributor, their respective affiliates, and/or the UnUniFi Project Contributors shall be responsible for or liable for the value of GUU, the transferability and/or liquidity of GUU and/or the availability of any market for GUU through third parties or otherwise; and\\
(e)	you acknowledge, understand and agree that you are not eligible to participate in the distribution of GUU if you are a citizen, national, resident (tax or otherwise), domiciliary and/or green card or permanent visa holder of a geographic area or country (i) where it is likely that the distribution of GUU would be construed as the sale of a security (howsoever named), financial service or investment product and/or (ii) where participation in token distributions is prohibited by applicable law, decree, regulation, treaty, or administrative act (including without limitation the United States of America, Canada, and the People's Republic of China); and to this effect you agree to provide all such identity verification document when requested in order for the relevant checks to be carried out.\\
The Company, the Distributor and the UnUniFi Project Contributors do not and do not purport to make, and hereby disclaims, all representations, warranties or undertaking to any entity or person (including without limitation warranties as to the accuracy, completeness, timeliness, or reliability of the contents of the Token Documentation or the Website, or any other materials published by the Company or the Distributor). To the maximum extent permitted by law, the Company, the Distributor, their respective affiliates and service providers shall not be liable for any indirect, special, incidental, consequential or other losses of any kind, in tort, contract or otherwise (including, without limitation, any liability arising from default or negligence on the part of any of them, or any loss of revenue, income or profits, and loss of use or data) arising from the use of the Token Documentation or the Website, or any other materials published, or its contents (including without limitation any errors or omissions) or otherwise arising in connection with the same. Prospective acquirors of GUU should carefully consider and evaluate all risks and uncertainties (including financial and legal risks and uncertainties) associated with the distribution of GUU, the Company, the Distributor and the UnUniFi Project Contributors.\\
GUU Token: GUU are designed to be utilised, and that is the goal of the GUU distribution. In particular, it is highlighted that GUU:\\
(a)	does not have any tangible or physical manifestation, and does not have any intrinsic value (nor does any person make any representation or give any commitment as to its value);\\
(b)	is non-refundable and cannot be exchanged for cash (or its equivalent value in any other digital asset) or any payment obligation by the Company, the Distributor or any of their respective affiliates;\\
(c)	does not represent or confer on the token holder any right of any form with respect to the Company, the Distributor (or any of their respective affiliates), or their revenues or assets, including without limitation any right to receive future dividends, revenue, shares, ownership right or stake, share or security, any voting, distribution, redemption, liquidation, proprietary (including all forms of intellectual property or licence rights), right to receive accounts, financial statements or other financial data, the right to requisition or participate in shareholder meetings, the right to nominate a director, or other financial or legal rights or equivalent rights, or intellectual property rights or any other form of participation in or relating to the UnUniFi Protocol, the Company, the Distributor and/or their service providers;\\
(d)	is not intended to represent any rights under a contract for differences or under any other contract the purpose or intended purpose of which is to secure a profit or avoid a loss;\\
(e)	is not intended to be a representation of money (including electronic money), payment instrument, security, commodity, bond, debt instrument, unit in a collective investment or managed investment scheme or any other kind of financial instrument or investment;\\
(f)	is not a loan to the Company, the Distributor or any of their respective affiliates, is not intended to represent a debt owed by the Company, the Distributor or any of their respective affiliates, and there is no expectation of profit nor interest payment; and\\
(g)	does not provide the token holder with any ownership or other interest in the Company, the Distributor or any of their respective affiliates.\\
Notwithstanding the GUU distribution, users have no economic or legal right over or beneficial interest in the assets of the Company, the Distributor, or any of their affiliates after the token distribution.\\
To the extent a secondary market or exchange for trading GUU does develop, it would be run and operated wholly independently of the Company, the Distributor, the distribution of GUU and the UnUniFi Protocol. Neither the Company nor the Distributor will create such secondary markets nor will either entity act as an exchange for GUU.\\
Informational purposes only: The information set out herein is only conceptual, and describes the future development goals for the UnUniFi Protocol to be developed. In particular, the project roadmap in the Token Documentation is being shared in order to outline some of the plans of the UnUniFi Project Contributors, and is provided solely for INFORMATIONAL PURPOSES and does not constitute any binding commitment. Please do not rely on this information in deciding whether to participate in the token distribution because ultimately, the development, release, and timing of any products, features or functionality remains at the sole discretion of the Company, the Distributor or their respective affiliates, and is subject to change. Further, the Token Documentation or the Website may be amended or replaced from time to time. There are no obligations to update the Token Documentation or the Website, or to provide recipients with access to any information beyond what is provided herein.\\
Regulatory approval: No regulatory authority has examined or approved, whether formally or informally, any of the information set out in the Token Documentation or the Website. No such action or assurance has been or will be taken under the laws, regulatory requirements or rules of any jurisdiction. The publication, distribution or dissemination of the Token Documentation or the Website does not imply that the applicable laws, regulatory requirements or rules have been complied with.\\
Cautionary Note on forward-looking statements: All statements contained herein, statements made in press releases or in any place accessible by the public and oral statements that may be made by the Company, the Distributor and/or the UnUniFi Project Contributors, may constitute forward-looking statements (including statements regarding the intent, belief or current expectations with respect to market conditions, business strategy and plans, financial condition, specific provisions and risk management practices). You are cautioned not to place undue reliance on these forward-looking statements given that these statements involve known and unknown risks, uncertainties and other factors that may cause the actual future results to be materially different from that described by such forward-looking statements, and no independent third party has reviewed the reasonableness of any such statements or assumptions. These forward-looking statements are applicable only as of the date indicated in the Token Documentation, and the Company, the Distributor as well as the UnUniFi Project Contributors expressly disclaim any responsibility (whether express or implied) to release any revisions to these forward-looking statements to reflect events after such date.\\
References to companies and platforms: The use of any company and/or platform names or trademarks herein (save for those which relate to the Company, the Distributor or their respective affiliates) does not imply any affiliation with, or endorsement by, any third party. References in the Token Documentation or the Website to specific companies and platforms are for illustrative purposes only.\\
English language: The Token Documentation and the Website may be translated into a language other than English for reference purpose only and in the event of conflict or ambiguity between the English language version and translated versions of the Token Documentation or the Website, the English language versions shall prevail. You acknowledge that you have read and understood the English language version of the Token Documentation and the Website.\\
No Distribution: No part of the Token Documentation or the Website is to be copied, reproduced, distributed or disseminated in any way without the prior written consent of the Company or the Distributor. By attending any presentation on this Token Documentation or by accepting any hard or soft copy of the Token Documentation, you agree to be bound by the foregoing limitations.\\


\subsection{Risk of Losing GUU due to Loss of Private Key}
The private key itself or a combination of private key shall be necessary for the disposal of the User's GUU, and the management of the private key shall be managed under the User's own authority and responsibility. 
The loss of the private keys associated with the wallet in which the user's GUU is stored is the same as the loss of the GUU itself. 
Phishing attacks against you or the GUU on your device may result in loss of GUU due to malware attacks, DoS attacks, consensus-based attacks, or any other form of attack.

\subsection{Risks Related to the UnUniFi Protocol}
Since GUU is based on the UnUniFi Protocol any malfunction, failure or failure of the UnUniFi Protocol may have a material adverse effect on GUU and may render GUU temporarily unusable.

\subsection{Risk of mining attacks}
GUU, like other distributed cryptographic tokens based on public chain protocols, may be subject to mining attacks during the verification of token transactions on the blockchain. 
These attacks may pose a risk to the recording of transactions related to GUU.

\subsection{Changes in Laws and Regulations and Taxation Risk}
There may be future changes in laws, government ordinances, statutes, regulations, orders, notices, guidelines, or other regulations or taxation systems related to GUU. 
You are responsible for making your own decisions regarding the taxation of the GUU. 

\subsection{Risks Due to Input Errors and Other Factors by User}
There is a risk of unintended transaction results due to input errors or any other actions by the User, failure, malfunction or operational status of the User's or a third party's communication or system equipment, natural disasters, cyber attacks or any other causes. 

\subsection{Relationship between Users}
Any transactions, communications, disputes, etc., arising between users and other users or third parties in relation to the Company's website shall be the responsibility of the users. 

\section{Contributions}
We have already made some little contributions to the Cosmos ecosystems. \\
https://github.com/cosmos-client/cosmos-client-ts 

\section{Contact}
To contact us on the UnUniFi topic, please create an issue ticket in GitHub. \\
https://github.com/UnUniFi


\end{document}
